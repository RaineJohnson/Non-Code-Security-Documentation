\documentclass[letterpaper,11pt]{article}

\usepackage{latexsym}
\usepackage[empty]{fullpage}
\usepackage{titlesec}
\usepackage{marvosym}
\usepackage[usenames,dvipsnames]{color}
\usepackage{verbatim}
\usepackage{enumitem}
\usepackage[colorlinks = true,
            linkcolor = red,
            urlcolor  = blue,
            citecolor = blue,
            anchorcolor = blue]{hyperref}
\usepackage{fancyhdr}
\usepackage[english]{babel}
\usepackage{tabularx}
\usepackage{fontawesome5}
\usepackage{multicol}
\usepackage{comment}
\usepackage{dirtytalk}
\setlength{\multicolsep}{-3.0pt}
\setlength{\columnsep}{-1pt}
\input{glyphtounicode}


%----------FONT OPTIONS----------
% sans-serif
% \usepackage[sfdefault]{FiraSans}
% \usepackage[sfdefault]{roboto}
% \usepackage[sfdefault]{noto-sans}
% \usepackage[default]{sourcesanspro}

% serif
% \usepackage{CormorantGaramond}
% \usepackage{charter}

\hypersetup{pdfborder = 0 0 0}
\pagestyle{fancy}
\fancyhf{} % clear all header and footer fields
\fancyfoot{}
\renewcommand{\headrulewidth}{0pt}
\renewcommand{\footrulewidth}{0pt}

% Adjust margins
\addtolength{\oddsidemargin}{-0.6in}
\addtolength{\evensidemargin}{-0.5in}
\addtolength{\textwidth}{1.19in}
\addtolength{\topmargin}{-.7in}
\addtolength{\textheight}{1.4in}

\urlstyle{same}

\raggedbottom
\raggedright
\setlength{\tabcolsep}{0in}

% Sections formatting
\titleformat{\section}{
  \vspace{-4pt}\scshape\raggedright\large\bfseries
}{}{0em}{}[\color{black}\titlerule \vspace{-5pt}]

% Ensure that generate pdf is machine readable/ATS parsable
\pdfgentounicode=1

%-------------------------
% Custom commands
\newcommand{\resumeItem}[1]{
  \item\small{
    {#1 \vspace{-2pt}}
  }
}

\newcommand{\classesList}[4]{
    \item\small{
        {#1 #2 #3 #4 \vspace{-2pt}}
  }
}

\newcommand{\resumeSubheading}[4]{
  \vspace{-2pt}\item
    \begin{tabular*}{1.0\textwidth}[t]{l@{\extracolsep{\fill}}r}
      \textbf{#1} & \textbf{\small #2} \\
      \textit{\small#3} & \textit{\small #4} \\
    \end{tabular*}\vspace{-7pt}
}

\newcommand{\resumeSubSubheading}[2]{
    \item
    \begin{tabular*}{0.97\textwidth}{l@{\extracolsep{\fill}}r}
      \textit{\small#1} & \textit{\small #2} \\
    \end{tabular*}\vspace{-7pt}
}

\newcommand{\resumeProjectHeading}[2]{
    \item
    \begin{tabular*}{1.001\textwidth}{l@{\extracolsep{\fill}}r}
      \small#1 & \textbf{\small #2}\\
    \end{tabular*}\vspace{-7pt}
}

\newcommand{\resumeSubItem}[1]{\resumeItem{#1}\vspace{-4pt}}

\renewcommand\labelitemi{$\vcenter{\hbox{\tiny$\bullet$}}$}
\renewcommand\labelitemii{$\vcenter{\hbox{\tiny$\bullet$}}$}

\newcommand{\resumeSubHeadingListStart}{\begin{itemize}[leftmargin=0.0in, label={}]}
\newcommand{\resumeSubHeadingListEnd}{\end{itemize}}
\newcommand{\resumeItemListStart}{\begin{itemize}}
\newcommand{\resumeItemListEnd}{\end{itemize}\vspace{-5pt}}



\begin{document}

\begin{center}
    {\Huge \scshape Raine Johnson} \\ \vspace{3pt}
    \small
    \raisebox{-0.1\height}\faPhone\ +1 (425) 273-1772 ~
    \href{mailto:rainewow@gmail.com}{\faEnvelope\ rainewow@gmail.com} ~ 
    \href{https://www.linkedin.com/in/raine-johnson}{\faLinkedin\ linkedin.com/in/raine-johnson}  ~
    \href{https://github.com/rainejohnson}{\faGithub\ rainejohnson}
    \vspace{-8pt}
\end{center}

\begin{comment}
% -----------Summary-----------
\section{Objective}
  \small{Enter your objective here}
\vspace{-12pt}
% adjust vscape based on the space you want to leave between each section.
\end{comment}

%-----------EDUCATION-----------
\section{Education}
  \resumeSubHeadingListStart
        \resumeSubheading
      {B.S. Computer Science and Cybersecurity}{Sept 2022 -- Jan 2026}
      {University of Washington Bothell}{}

      %leave a line after this
   
  \resumeSubHeadingListEnd
\vspace{-18pt}


%-----------EXPERIENCE-----------
\section{Experience}
  \resumeSubHeadingListStart
  
    \resumeSubheading
      {The Estee Lauder Companies}{June 2025 -- Aug 2025}
      {Operational Technology Cybersecurity Analyst Intern}{New York City}
      \resumeItemListStart
        \resumeItem { Deployed and configured Dispel infrastructure across global OT systems supporting a company with over 62,000 employees and operations in approximately 150 countries, ensuring secure remote vendor access and reducing third-party exposure.}
        \resumeItem {Onboarded and optimized Armis for real‑time asset visibility and threat detection across OT/IoT networks, aligning with Estee Lauder’s robust cybersecurity framework that protects \$15.6B in annual revenue and 56\% in skincare business share.}
        \resumeItem {Collaborated with cross-functional IT and risk teams to integrate cybersecurity tools into industrial networks, supporting compliance measures and safeguarding critical infrastructure.}
        \resumeItemListEnd
      
    \resumeSubheading
      {Bose}{Jan 2022 -- Sept 2022}
      {Roadshow Engineer}{International}
      \resumeItemListStart
        \resumeItem {Collaborated with a team to manage sensitive project files in a secure git repository, ensuring proper version control and access security.}
        \resumeItem {Designed virtual reality (VR) demonstrations using C++ with a focus on securing communication channels between hardware systems during international presentations.}
        \resumeItem {Ensured systems operated securely and in line with both marketing and engineering requirements, adhering to data protection and confidentiality standards.}
        \resumeItemListEnd

    \resumeSubheading
      {Microsoft}{July 2021 -- Sept 2022}
      {Certification and Accessibility Tester}{Redmond}
      \resumeItemListStart
        \resumeItem {Conducted security tests on Xbox products, including network calls and crash dumps, identifying potential vulnerabilities and enhancing product integrity.}
        \resumeItem {Introduced and enforced strict security protocols for protecting personal identifiable information (PII) during testing.}
        \resumeItem {Utilized debugging tools to identify and resolve security issues, including title stability and system integrity, contributing to the overall resilience of Xbox products.}
%Based on the number of point you want to add the "\resumeItem" for each point you want to add.
    \resumeItemListEnd
    
  \resumeSubHeadingListEnd
\vspace{-16pt}
% adjust vscape based on the space you want to leave between each section.


\begin{comment}
%------------Awards-----------
\section{Awards}
  \resumeSubHeadingListStart
    \resumeSubheading
      {Name}{Date}
      {}{}
      \small{Description.}
    
  \resumeSubHeadingListEnd
\vspace{-16pt}

\end{comment}

%-----------PROJECTS-----------
\section{Projects}
    \vspace{-6pt}
    \resumeSubHeadingListStart
    \resumeProjectHeading
          {\textbf{Secure Home Lab Network Architecture \& Hardening} $|$ \emph{pfSense, Proxmox, Suricata, ntopng \href{https://github.com/RaineJohnson}{}}}{}
          \resumeItemListStart
            \resumeItem{Built a layered home lab network on Proxmox with pfSense as the perimeter firewall/router, segmented into multiple VLANs using Ubiquiti UniFi switches and Access Points to emulate enterprise segmentation.}
            \resumeItem{Deployed Suricata IDS/IPS and ntopng for real-time traffic analysis and threat detection, tuning signatures and flow monitoring to identify anomalous behavior across internal and external network traffic.}
            \resumeItem{Integrated OpenVPN and WireGuard tunnels for secure remote access into the home lab, with automated VPN key management and strict access policies to isolate administrative traffic.}
          \resumeItemListEnd
          \vspace{-14pt}

    \resumeProjectHeading
          {\textbf{Custom Password Manager \& Credential Security App} $|$ \emph{Python, Linux CLI, AES-CBC, SHA-256, \href{https://github.com/RaineJohnson/PersonalPasswordManager.git}{Github}}}{}
          \resumeItemListStart
            \resumeItem{Developed a CLI-based password manager in Python supporting encrypted storage and management of multiple credential records within a local database.}
            \resumeItem{Implemented AES-CBC encryption at rest using PyCryptodome and SHA-256 hashing for master password verification, preventing plaintext credential exposure.}
            \resumeItem{Built secure credential lifecycle features, including password generation, rotation, backup, and database erasure with authentication enforcement.}
          \resumeItemListEnd
          \vspace{-14pt}
    \resumeSubHeadingListEnd
\vspace{-3pt}

%-----------Technical Skills-----------
\section{Skills and Tools}
 \begin{itemize}[leftmargin=0.1in, label={}]
    \small{\item{
     {Armis, Dispel, Cisco ISE, Palo Alto Firewalls, Illumio, pfSense, VLANs, network segmentation, packet filtering, Wireshark, Nmap, Nessus/Tenable, Splunk SIEM, SentinelOne, Metasploit (lab use), Kali Linux, Ubuntu, ServiceNow, Power BI, incident detection and response, access control, least-privilege enforcement, secure credential handling, risk analysis, NIST Cybersecurity Framework, Python, Java, JavaScript, SQL, HTML/CSS, Node.js (basic), secure scripting, encryption and hashing, Git, GitHub, virtualization (Oracle VM / VirtualBox), traffic analysis, VPN fundamentals, vulnerability scanning, threat hunting fundamentals, secure logging and monitoring, web security fundamentals (CSRF, injection risks).} \\
     
    }}
 \end{itemize}
 \vspace{-20pt}

\begin{comment}
%------------Publications-----------
\section{Publication}
\vspace{-6pt}
    \resumeSubHeadingListStart
    \resumeProjectHeading
    {\textbf{The Title of your paper} $|$ 
    \emph{\href{https://www.researchgate.net}{Publication link}}}{Month Year}
    \resumeItemListStart
            \resumeItem{Write about your paper.}
            \resumeItem{Write about your paper.}
          \resumeItemListEnd
    \resumeSubHeadingListEnd
\vspace{-16pt}
% adjust vscape based on the space you want to leave between each section.
\end{comment}

%-----------PROGRAMMING SKILLS-----------
\section{Certifications}
\textbf{Certificate 1} {: ISC2 Certified in Cybersecurity Certification} \\
\textbf{Certificate 2} {: Codecademy Cybersecurity Certification} \\
\textbf{Certificate 3} {: TryHackMe Security Analyst Certification}



\end{document}
